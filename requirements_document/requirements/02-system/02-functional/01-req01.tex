

\paragraph{}
\begin{description}
  \item [Statement] 
    \textit{}
  \item [Priority] \textit{A}
\end{description}

\paragraph{}
\begin{description}
\item [Aussage] \textit{Benutzer können sich mit Name, Email-Adresse und Passwort registrieren.}
\item [Priorität] \textit{A}
\end{description}

\paragraph{}
\begin{description}
  \item [Statement] 
    \textit{Die Email-Adresse muss bestätigt werden.}
  \item [Priority] \textit{B}
\end{description}


\paragraph{}
\begin{description}
\item[Aussage] \textit{Es gibt mindestens einen Administrator mit erweiterten Rechten. Er kann den erlaubten Speicherplatz und Rechenzeit der anonymen Nutzer festlegen. Er hat Zugriff auf alle Pakete.}
\item[Priorität] \textit{A}
\end{description}

\paragraph{}
\begin{description}
\item[Aussage] \textit{Angemeldete Nutzer können Gruppen bilden um gemeinsam an einem Datensatz/Paket zu arbeiten. Anonyme Nutzer können nicht zu Gruppen hinzugefügt werden. Jeder Nutzer einer Gruppe hat zugriff auf das komplette Paket.}
\item[Priorität] \textit{A}
\end{description}


\paragraph{}
\begin{description}
\item[Aussage] \textit{Die Algorithmen werden für den vorliegenden Datensatz vorsortiert, nur anwendbare werden angezeigt. Dementsprechend müssen die Daten und Algorithmen so gekennzeichnet werden, dass das System erkennt welche Algorithmen auf welche Daten anwendbar sind.}
\item[Priorität] \textit{A}
\end{description}

\paragraph{}
\begin{description}
\item[Aussage] \textit{Nutzer haben die möglichkeit eigene Algorithmen als .jar datei zum Programm hinzu zu fügen.
Diese müssen dann noch von einem Administrator für alle Nutzer freigeschaltet werden.}
\item[Priorität] \textit{A}
\end{description}


