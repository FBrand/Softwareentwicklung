\paragraph{}
\begin{description}
\item [Aussage] \textit{Es wird ein Online-Framework für Datamining und Machinelearning entwickelt.}
\item [Priorität] \textit{A+}
\end{description}

\paragraph{}
\begin{description}
\item [Aussage] \textit{Benutzer können sich registrieren oder das Programm anonym benutzen.}
\item [Priorität] \textit{A}
\end{description}

\paragraph{}
\begin{description}
\item[Aussage] \textit{Anonyme Nutzer haben eingeschränkte Rechte.}
\item[Priorität] \textit{A}
\end{description}

\paragraph{}
\begin{description}
\item[Aussage] \textit{Es gibt mindestens einen Administrator mit erweiterten Rechten.}
\item[Priorität] \textit{A}
\end{description}

\paragraph{}
\begin{description}
\item[Aussage] \textit{Angemeldete Nutzer können Gruppen bilden um gemeinsam an einem Datensatz/Paket zu arbeiten.}
\item[Priorität] \textit{A}
\end{description}

\paragraph{}
\begin{description}
\item[Aussage] \textit{Die Benutzer sind lokal auf dem Server gespeichert, auf einem anderen Server muss man sich neu anmelden.}
\item[Priorität] \textit{B}
\end{description}

\paragraph{}
\begin{description}
\item[Aussage] \textit{Das Programm arbeitet mit in Java implementierten Algorithmen.}
\item[Priorität] \textit{A}
\end{description}

\paragraph{}
\begin{description}
\item[Aussage] \textit{Die Algorithmen dienen der Kassifikation und evtl. Clustering.}
\item[Priorität] \textit{A}
\end{description}

\paragraph{}
\begin{description}
\item[Aussage] \textit{Die Algorithmen SMO, J48 und Random Forest sind bereits implementiert.}
\item[Priorität] \textit{A}
\end{description}

\paragraph{}
\begin{description}
\item[Aussage] \textit{Laufende Algorithmen können abgebrochen werden.}
\item[Priorität] \textit{A}
\end{description}

\paragraph{}
\begin{description}
\item[Aussage] \textit{Die Algorithmen werden für den vorliegenden Datensatz vorsortiert, nur anwendbare werden angezeigt.}
\item[Priorität] \textit{A}
\end{description}

\paragraph{}
\begin{description}
\item[Aussage] \textit{Nutzer haben die Möglichkeit eigene Algorithmen als .jar datei zum Programm hinzu zu fügen.}
\item[Priorität] \textit{A}
\end{description}

\paragraph{}
\begin{description}
\item[Aussage] \textit{Es steht ein festes Interface für die Algorithmen zur Verfügung.}
\item[Priorität] \textit{A}
\end{description}

\paragraph{}
\begin{description}
\item[Aussage] \textit{Algorithmen werden Vom Admin geprüft, bevor sie bereitgestellt werden.}
\item[Priorität] \textit{B}
\end{description}

\paragraph{}
\begin{description}
\item[Aussage] \textit{Das Programm steht über eine Web-Oberfläche zur Verfügung. Eine gesonderte mobile Website ist nicht gefordert.}
\item[Priorität] \textit{A}
\end{description}

\paragraph{}
\begin{description}
\item[Aussage] \textit{Die Website steht in englischer Sprache zur Verfügung.}
\item[Priorität] \textit{A}
\end{description}

\paragraph{}
\begin{description}
\item[Aussage] \textit{Die Website wird in HTML programmiert.}
\item[Priorität] \textit{A}
\end{description}

\paragraph{}
\begin{description}
\item[Aussage] \textit{Dem Nutzer steht eine Übersicht über alle Pakete zur Verfügung, auf die er Zugriff hat.}
\item[Priorität] \textit{A}
\end{description}

\paragraph{}
\begin{description}
\item [Aussage] \textit{Die Modelle werden in einer RDF-Datenbank gespeichert.}
\item [Priorität] \textit{A}
\end{description}

\paragraph{}
\begin{description}
\item[Aussage] \textit{Die Datenbank unterstützt SPARQL.}
\item[Priorität] \textit{A}
\end{description}

\paragraph{}
\begin{description}
\item[Aussage] \textit{Der Zugriff auf das System ist sicher. Niemand kann Daten, für die er kein Zugriffsrecht besitzt, manipulieren.}
\item[Priorität] \textit{A}
\end{description}

\paragraph{}
\begin{description}
\item[Aussage] \textit{Mithilfe der Algorithmen sollen aus den Daten Modelle erstellt und gespeichert werden können (in RDF-Format.}
\item[Priorität] \textit{A}
\end{description}

\paragraph{}
\begin{description}
\item[Aussage] \textit{Vorhandene Modelle sollen auf andere Daten angewendet werden können und dadurch Vorhersagen machen oder die Güte des Modells bewerten.}
\item[Priorität] \textit{A}
\end{description}

\paragraph{}
\begin{description}
\item[Aussage] \textit{Alle Daten sollen auf einem Server gespeichert werden.}
\item[Priorität] \textit{A}
\end{description}

\paragraph{}
\begin{description}
\item[Aussage] \textit{Die Daten sollen von den Nutzern in die Datenbank geladen werden können.(in arff bzw. csv Format.}
\item[Priorität] \textit{A}
\end{description}


\paragraph{}
\begin{description}
\item[Aussage] \textit{Der Nutzer kann seperate Datenbestände zum Lernen und Testen hochladen, oder einen Datenbestand der auf dem Server in 2 Teile geteilt wird(Verhältnis vom Nutzer festzulegen).}
\item[Priorität] \textit{A}
\end{description}

\paragraph{}
\begin{description}
\item[Aussage] \textit{Daten anonymer Nutzer werden nach einer vom Admin festgelegten Frist vom Server gelöscht.}
\item[Priorität] \textit{A}
\end{description}

\paragraph{}
\begin{description}
\item[Aussage] \textit{Die Daten werden kategorisiert gespeichert.}
\item[Priorität] \textit{A}
\end{description}


\paragraph{}
\begin{description}
\item[Aussage] \textit{Daten anonymer Nutzer werden in einem gemeinsamen Anonym-Paket gespeichert.}
\item[Priorität] \textit{A}
\end{description}

\paragraph{}
\begin{description}
\item[Aussage] \textit{Einmal erstellte Modelle sollen nicht mehr verändert werden können.}
\item[Priorität] \textit{A}
\end{description}

\paragraph{}
\begin{description}
\item[Aussage] \textit{Es sollen auch Modelle von Nutzern hochgeladen werden können.}
\item[Priorität] \textit{B}
\end{description}

\paragraph{}
\begin{description}
\item[Aussage] \textit{Wird Versucht ein Modell zu erstellen, dass bereits vorhanden ist, soll gefragt werden, ob stattdessen das bereits berechnete Modell aufgerufen werden soll.}
\item[Priorität] \textit{B}
\end{description}

\paragraph{}
\begin{description}
\item [Aussage] \textit{Das System soll Pakete anbieten. Jedes Paket soll Datensätze, Modelle und Ergebnisse enthalten.}
\item [Priorität] \textit{A}
\end{description}

\paragraph{}
\begin{description}
\item[Aussage] \textit{Jeder Benutzer soll Pakete erstellen können.}
\item[Priorität] \textit{A}
\end{description}

\paragraph{}
\begin{description}
\item[Aussage] \textit{Jedes Paket ist über seine ID such- und adressierbar.}
\item[Priorität] \textit{A}
\end{description}

\paragraph{}
\begin{description}
\item[Aussage] \textit{Die Pakete sollen Zurgriffsrechte haben. Standard ist Privat.}
\item[Priorität] \textit{A}
\end{description}

\paragraph{}
\begin{description}
\item[Aussage] \textit{Der Benutzer soll die Zugriffsrechte des Pakets ändern können. z.B. einer anderen Gruppe Schreibrechte auf ein bestimmtes Paket geben.}
\item[Priorität] \textit{A}
\end{description}

\paragraph{}
\begin{description}
\item[Aussage] \textit{Der Benutzer kann Pakete öffentlich machen.}
\item[Priorität] \textit{A}
\end{description}

\paragraph{}
\begin{description}
\item[Aussage] \textit{Nach dem Bearbeiten der Datensätze soll der Kunde die Ergebnisse kriegen. Die Ergebnisse sollen selektier- und formatierbar sein.}
\item[Priorität] \textit{A}
\end{description}

\paragraph{}
\begin{description}
\item[Aussage] \textit{Der Kunde soll die Ergebnisse herunterladen können. Der Kunde soll wählen können, ob er die Ergebnisse in HTML oder PDF herunterladen möchte.}
\item[Priorität] \textit{A}
\end{description}
